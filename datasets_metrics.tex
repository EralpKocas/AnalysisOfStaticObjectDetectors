\documentclass{article}

\begin{document}

\setlength{\parindent}{6ex}

\indent

In this section datasets and performance metrics will be explained. \par 
Datasets are used in training and testing of detectors. Datasets consist of some 
specific type of objects, in object detection case they are images, and the 
corresponding ground-truth information about the categories in images that 
detector performs to detect. So, the mainly used datasets are Pascal VOC and 
MS COCO. \par

Performance metrics are the way to measure the accuracy of detectors. In object 
detection, improving precision and recall yields to better test performance. The 
difference lies under the calculation of these metrics for object detection since 
correctly classifying the object in the image is not enough, detector also correctly 
localize the object in the image. So, to measure the correctness of each detection, 
a metric called Intersection over Union is used:
\begin{itemize}
    \item Intersection over Union (IoU)
    IoU is the ratio between the intersection of predicted and ground-truth bounding box 
    divided by union of predicted and ground-truth bounding box. 
    \item Correctness of Detection
    IoU is used as a threshold to identify a detection as positive or negative. The 
    most commonly used IoU threshold is 0.5 which means if IoU value of a detection is 
    bigger than 0.5, that detection is counted as True Positive; otherwise, False Positive.
    \item The following concepts are used by the metrics:
    \begin{itemize}
        \item True Positive (TP): A correct detection, IoU $\geq$ threshold. 
        \item False Positive (FP): A wrong detection, IoU $\leq$ threshold.
        \item False Negative (FN): A ground-truth not detected.
        \item True Negative (TN): Since there are many possible bounding boxes that 
        do not contain any object, counting corrected misdetection is not necessarily 
        needed.
    \end{itemize}
    \item These calculated concepts above are used to measure two main metrics: precision and 
    recall. Precision is the ratio of true positives to the sum of true positives and false 
    positives. Recall is the ratio of true positives to the sum of true positives and false 
    negatives.
    \item Lastly, the performance indicator for detectors is mean average precision. Mean 
    average precision can be calculated by calculating the area under the precision-recall curve.
\end{itemize}
\end{document}