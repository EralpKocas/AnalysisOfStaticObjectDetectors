\documentclass{article}

\begin{document}

\setlength{\parindent}{6ex}

\indent

CornerNet \cite{cornernetcite} is a new approach to object detection. It detects objects using 
paired key points which means detecting bounding boxes using top-left and bottom-right 
corners of objects. In addition to this new approach, corner pooling is introduced. 
Corners are better localized using corner pooling. Using paired keypoints eliminates 
the need for using anchor boxes. \par

There are two drawbacks of using anchor boxes: 
\begin{enumerate}
    \item Anchor boxes require a huge set. Since most of the anchor boxes do not 
    overlap with ground-truth bounding boxes, a huge imbalance between positive and 
    negative anchor boxes is created and handling this imbalance slows down training.
    \item Anchor boxes introduce many hyperparameters and design choices to make, 
    it may even introduce more if a single network makes separate predictions at 
    multiple resolutions.
\end{enumerate}
\indent

Corner pooling is applied by taking the maximum values in two directions (horizontal 
and vertical) for each channel and adding these two values together. \par

In CornerNet, a single neural network predicts a heatmap for the top-left corners 
of all instances of the same object category, a heatmap for all bottom-right 
corners, and an embedding vector for each detected corner \cite{cornernetcite}.

\end{document}