\documentclass{article}

\begin{document}

\setlength{\parindent}{6ex}

\indent

The Faster R-CNN with FPN, SSD for 512 input's short dimension resolution, and YOLOv3 for 320 and 416 
input's short dimension resolution are analyzed with the MOT dataset and the results can be observed 
in table \ref{table:3}. The train set of MOT dataset is used as a test set on these models since 
their ground-truth values are provided. So, these models are not trained on the MOT dataset. 
They only trained on the COCO dataset. The class score threshold is a threshold for an object 
is accepted as its founded class. For instance, if SSD detects an object as a pedestrian 
with a class score of 0.85 and if the threshold is equal to 0.5, then, this object will be 
accepted as pedestrian and it will be included in the detected objects' list. However, if 
the prediction score is 0.3, then, the prediction will be rejected. The threshold for 
the COCO dataset is 0.5. So, to understand the effect of threshold on detection, YOLOv3 is 
analyzed on the MOT dataset for different threshold values: 0.0, 0.3, 0.5 as can be observed 
in table \ref{table:3}. Also, a selected image is visualized with detected bounding boxes 
on figure \ref{fig:exp_imgs} for different threshold values: 0.0, 0.3, 0.5, 0.7, 0.9. \par

The results in table \ref{table:3} show us that the performances of the analyzed detectors 
are compatible with their performances in table \ref{table:2}. In addition, in table \ref{table:3}, 
you can see the effect of input resolution and threshold. \par 

Reducing the resolution of the input image may lead to a loss of information. Therefore, using a bigger input 
image resolution may increase the average precision of detectors. As you can see in table \ref{table:3}, 
the average precision is increased from 0.4642 to 0.5236 when you use YOLOv3 with 416 input image 
resolution instead of 320 input image resolution. This change increases the number of true positives and 
false positives, however, the number of false negatives is reduced. Therefore, recall is increased 
significant amount which means a performance increase in detectors. \par

Reducing the threshold for class prediction may lead to an increase in the number of true positives. Yet, 
you can notice that it also leads to a huge increase in false positives. The reason for this increase can 
be observed in figure \ref{fig:exp_imgs}. You can see the difference in the number of detected objects in 
given thresholds for the given image. For instance, having a 0.0 threshold leads our detector to accept a detection 
as a true detection even the score of classification is around 0.02. Therefore, detectors predict so many 
detections. The numbers of true and false positives and false negatives  can be observed in table \ref{table:3}.
You can notice that the number of true positives reduces from 0.0  to 0.9 threshold but the number of false 
positives also reduces drastically. In addition, the number of false negatives increases in the corresponding order. 
The reason for this flow is that having a lower threshold leads to detect every part of the given image even though 
there is no object present. So, in the end, the detector covers every object of the image even with the parts that have 
no object. Therefore, all the objects are detected and TP is close to GT. Thus, FN is less. Also, since the detector 
detects many backgrounds as positive, the FP is also high. As a result, keeping the threshold low leads to a better 
average precision, yet, the precision reduces significant amount and the recall increases significant amount. This means 
that our detector is good at finding positives but our detector is not accurate since it detects 
almost anything as positive.  

\end{document}