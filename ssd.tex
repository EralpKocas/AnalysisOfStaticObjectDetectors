\documentclass{article}

\begin{document}

\setlength{\parindent}{6ex}

\indent

Single Shot MultiBox Detector is a single deep neural network which aims to detect 
objects in real-time. The main improvement in speed comes from eliminating region 
proposals as Faster R-CNN does. Although eliminating region proposals increases 
the speed of detection, it reduces the accuracy significantly. Thus, three main 
improvements are applied to increase accuracy:
\begin{enumerate}
    \item Multi-Scale Feature Maps for Detection
    \begin{itemize}
        \item As mentioned in section 2.3, handling multi-scale is important feature 
for object detectors. It can reduce the mAP for detector's performance since objects 
of different sizes in given image cannot be detected well. To handle multi-scale detection, 
SSD uses multi-scale feature maps instead of using different sizes of input images. 
The size of feature maps decreases through the network's architecture. For larger feature maps,
SSD can detect smaller objects and for smaller feature maps, smaller objects can be detected.
    \end{itemize}
    \item Convolutional Predictors for Detection
    \begin{itemize}
        \item Small-size convolutional filters are used to make prediction for 
object detection. These convolutional filters are applied on extracted feature maps 
to compute both localization and class scores. Each filter computes a bounding box and 
corresponding class scores for each category.
    \end{itemize}
    \item Default Bounding Boxes and Aspect Ratios
    \begin{itemize}
        \item SSD divides its feature maps into a grid and each feature map cells are 
associated with a set of default bounding boxes. 
    \end{itemize}
\end{enumerate}

\end{document}