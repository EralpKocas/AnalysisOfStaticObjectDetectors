\documentclass{article}

\begin{document}

\setlength{\parindent}{6ex}

\indent

Deconvolutional Single Shot Detector aims to contribute a new approach for object 
detection. The major changes in DSSD are to change the base network from VGG16 to 
ResNet101, using prediction modules, and adding deconvolution layers after convolutional 
layers of SSD. So, these changes will be examined as follows: 
\begin{enumerate}
    \item ResNet101
    \begin{itemize}
        \item The aim of changing base network from VGG16 to ResNet101 is to 
increase accuracy. However, it does not improve accuracy by itself. That's why, 
the prediction module is used to increase performance.
    \end{itemize}
    \item Prediction Module
    \begin{itemize}
        \item Due to the principle of improving sub-network can improve accuracy, 
original SSD approach for prediction is replaced with a prediction module which 
consists of a residual block. Thus, using ResNet101 with prediction module performs 
better than VGG16.
    \end{itemize}
    \item Deconvolution Layers
    \begin{itemize}
        \item The aim of using deconvolutional layers is to include more high-level 
context in detection, so that, deconvolution module integrates information from both 
earlier feature maps and earlier deconvolution layers. 
    \end{itemize}
\end{enumerate}

\end{document}